\section{Oscillation Theory and the PMNS Matrix}
In 1968, Bruno Pontecorvo suggested a process, known as \emph{neutrino oscillation}, by which neutrinos could change flavors \cite{Pontecorvo-Oscillations}.
The theory of neutrino oscillations was further developed for the neutrino sector by Ziro Maki, Masami Nakagawa and Shoichi Sakata in 1962 \cite{Maki-Nakagawa-Sakata}.

\subsection{The PMNS Mixing Matrix}
We now understand there to be three distinct flavors of neutrinos.
Neutrinos are created via the weak force as pure flavor eigenstates. 
These states are coherent superpositions of mass eigenstates.
Specifically, there exist three weak eigenstates of the left-handed neutrino fields that are related to three known neutrino mass eigenstates via the Pontecorvo-Maki-Nakagawa-Sakata (\emph{PMNS}) lepton mixing matrix.

\begin{equation}
\begin{pmatrix} \nu_e \\ 	\nu_\mu \\	\nu_\tau \end{pmatrix} = 
U_{PMNS} \begin{pmatrix} \nu_e \\ 	\nu_\mu \\	\nu_\tau \end{pmatrix} = 
\begin{pmatrix}
 U_{e 1} & U_{e 2} & U_{e 3} \\
 U_{\mu 1} & U_{\mu 2} & U_{\mu 3} \\
 U_{\tau 1} & U_{\tau 2} & U_{\tau 3} \\
\end{pmatrix} 	
\begin{pmatrix} \nu_1 \\ 	\nu_2 \\	\nu_3 \end{pmatrix}
\label{eqn:3flavor_pmns}
\end{equation}

The flavor eigenstates ($\nu_e , \nu_\mu , \nu_\tau$) describe the fields of the left-handed neutrinos coupling via the weak charge to the electron, muon, and tau respectively.
The three mass states ($\nu_1, \nu_2, \nu_3$) represent the mass eigenstates.
The mixing between the two types of states is given by the unitary matrix, $U_{PMNS}$.
The mixing may be written in a shortened form

\begin{equation}
\nu_{\alpha}\left(x\right) = \sum_i U_{\alpha i} \nu_{i}\left(x\right)
\label{eqn:3flavor_short}
\end{equation}

where ${\alpha=e,\mu,\tau}$ and ${i=1,2,3}$. 

Neutrinos interact via the weak force and are created in flavor states ${e,\mu,\tau}$.
The neutrino produced in flavor state $\nu_\alpha$ exists in a superposition of the three mass eigenstates.

where ${\delta_{ij}}$ and ${\delta_{\alpha \beta}}$ are Kronecker delta functions.
As a 3 ${\times}$ 3 unitary matrix, the PMNS matrix may be parametrized in terms of three mixing angles and six phases.
Of these phases, five may be removed by rephasing the lepton fields with no change to the underlying physics, leaving one physical phase ralted to CP violation.

The PMNS may be written in terms of the product of three smaller unitary matrices using these mixing angles.

\begin{equation}
U_{PMNS} = 
\begin{pmatrix}
1 & 0 & 0 \\
0 & c_{23} & s_{23} \\
0 & -s_{23} & c_{23} \\
\end{pmatrix}
\begin{pmatrix}
c_{13} & 0 & s_{13} e^{-i\delta_{CP}} \\
0 & 1 & 0 \\
-s_{13} e^{i \delta_{CP}} & 0 & c_{13} \\
\end{pmatrix}
\begin{pmatrix}
c_{12} & s_{12} & 0 \\
-s_{12} & c_{12} & 0 \\
0 & 0 & 1 \\
\end{pmatrix}
\label{eqn:pmns_parametrized}
\end{equation}

where the notation ${c_{ij}}$ idenotes ${\cos\left(\theta_{ij}\right)}$ and ${s_{ij}}$ denotes for ${\sin\left(\theta_{ij}\right)}$.

Note that if neutrinos are Majorana fermions, the additional phases may not be removed without making the masses complex.
The Majorana terms form additional diagonal terms in Equation~\ref{eqn:pmns_parametrized}.
While Majorana mass terms are beyond the scope of this work, further information may be found in \cite{Review-PMNS,Review-MajoranaNu}.

The three submatrices of Equation~\ref{eqn:pmns_parametrized} have historically been studied by different types of experiments. 
This history has lead to the proliferation of alternative names for the matrices and of the mixing angles.

\begin{equation}
U_{PMNS} = U_{Atmospheric} U_{Reactor} U_{Solar}
\end{equation}

This leads to the alternative names of the mixing angles, with ${\theta_{23}}$, ${\theta_{13}}$, and ${\theta_{12}}$ being referred to as the atmospheric mixing angle, the reactor mixing angle, and the solar mixing angle respectively.

\label{subsec:vacuum}
\subsection{Neutrino Mixing in Vacuum}
Propagation of neutrinos requires the use of the Hamiltonian.
However, the flavor states are not eigenstates of the Hamiltonian.
For propagation of the neutrino, the mass eigenstates must instead be used.

\begin{equation}
\ket{\nu\left(t=0\right)} = \ket{\nu_\alpha} = \sum_i U_{\alpha i}^* \ket{\nu_i}
\end{equation}

The propagation leads to a neutrino state at time ${t\neq0}$ which is no longer a pure flavor state.

\begin{equation}
\ket{\nu\left(t\right)} = \sum_i U^*_{\alpha i} e^{-iE_i t} \ket{\nu_i}
\end{equation}

where ${E_i} = \sqrt{p^2+m_i^2}$ is the total energy of the ${i}$th mass eigenstate.
If the neutrino state interacts, the flavor eigenstate must again be used to calculate the probabilities of interacting as each of the three known flavors.

\begin{equation}
P\left(\nu_\alpha\rightarrow\nu_\beta\right) = \left|\bra{\nu_\beta}\ket{\nu\left(t\right)}\right|^2 
                 															= \left|\sum_i U_{\beta i} U^*_{\alpha i} e^{-i E_i t} \right|^2
\label{eqn:pmns_probability}
\end{equation}

Proper calculations from this point can be performed by treating each neturino as a quantum mechanical wave packet \cite{OscillationWavePackets}.
This allows for the full description of neutrino oscillation in the context of decoherence of the mass states during propagation, allowing each mass state to possess separate momenta.

In practice, the description of neutrino oscillations necessary for this work is adequately described by making a few simplifying assumptions.
In particular, this work assumes that all mass eigenstates propagate as plane waves possessing identical, well-defined momenta \cite{Review-PMNS}.
Neutrinos are further assumed to be extremely relativistic at the energies of interest, an assumption well-justified by cosmological fits to the sum of the three neutrino masses, which give an upper limit of around 0.2 eV \cite{PDG-2015}.
The total neutrino energy is also assumed to be unchanged during propagation.
The resulting calculation of the oscillation probabilities is identical in both the simplified version and the full derivation.

To begin, equation~\ref{eqn:pmns_probability} is expanded by explicitly including the complex conjugate,  
\begin{equation}
P\left(\nu_\alpha\rightarrow\nu_\beta\right) =  \sum_i^3 U^*_{\beta i} U_{\alpha i} \sum_j U_{\beta j} U^*_{\alpha j} e^{i \left(E_i-E_j\right) t} \ \ \ \alpha,\beta = e,\mu\tau.
\label{eqn:pmns_probability_expanded}
\end{equation}

In the highly relativistic limit, $E \gg m_{i}$, and $t \approx L$ where $L$ is the distance traveled during propation.
Using these two approximations, the exponential term in Equation~\ref{eqn:pmns_probability} may be rewritten using Euler's formula

\begin{equation}
 e^{i \left(E_i-E_j\right) t} = 1 - 2\sin^2\left(\frac{m_{ij}^2 L}{4E}\right) + i \sin\left(\frac{m_{ji}^2 L}{2E}\right)
\end{equation}

Note that a new shorthand has been defined, ${\Delta m^2_{ji} = m^2_j - m^2_i}$, giving a fundamental parameter of neutrino oscillations.
The PMNS terms of equation~\ref{eqn:pmns_probability_expanded} may be expanded further, yielding 

\begin{equation}
\left| \sum_j U_{\beta j} U^*_{\alpha j}\right|^2 = \delta_{\alpha \beta} + 2 \sum_{i<j} \sum_i U^*_{\beta i} U_{\alpha i}  U_{\beta j} U^*_{\alpha j}
\end{equation}

where the factor of two arises due to the symmetry ${i \leftrightarrow j}$.
Putting the terms together, the final oscillation probability formula is

\begin{equation}
\begin{aligned}
P\left(\nu_\alpha\rightarrow\nu_\beta\right) ={} &	 \delta_{\alpha \beta} -  4 \sum_{i<j} Re\left[ \sum_i U^*_{\beta i} U_{\alpha i}  U_{\beta j} U^*_{\alpha j} \right] sin^2\left(\frac{m_{ij}^2 L}{4E}\right) \\
	& + 2 \sum_{i<j} Im\left[ \sum_i U^*_{\beta i} U_{\alpha i}  U_{\beta j} U^*_{\alpha j} \right] sin\left(\frac{m_{ji}^2 L}{2E}\right).
\end{aligned}
\label{eqn:oscil_probabilities}
\end{equation}

This calculation has been derived for neutrinos.
To calculate the probabilities for anti-neutrinos, the calculation changes by replacing ${U \rightarrow U^*}$, resulting in a change in sign of the last term of Equation~\ref{eqn:oscil_probabilities}.

From Equation~\ref{eqn:oscil_probabilities}, the general form of the oscillation probabilities becomes clear. 
The PMNS matrix elements yield the amplitude of oscillations, while the phase of the oscillations is related to three quantities: the squared difference in the masses, ${\Delta m^2_{ji}}$; the baseline, or distance traveled, ${L}$; and the energy of the neutrinos.
Only one of these three is a fundamental physics parameter.
The choices of energy sensitivity and baseline are used to define characteristics of detectors  used for measurements of the various mass splitting parameters and oscillation mixing angles.

Note that the oscillation probability is insensitive to the sign of the mass splitting parameter.

\label{subsec:msw}
\subsection{Matter Effects in Oscillation}
Calculations up to this point have assumed neutrinos oscilating in vacuum.
Modifications required for a description of matter effects begin with a modification of the Hamiltonian with a potential, ${V}$, due to coherent forward scattering of neutrino on electrons and nucleons in the medium \cite{Review-MSW}.

\begin{equation}
H = H_0 + V
\end{equation}

The value of ${H_0}$ is the value of vacuum Hamiltonian.
In the two-flavor case, the Hamiltonian can be shown \cite{Review-PMNS,Review-PMNS2} to be

\begin{equation}
H_0 = \frac{\Delta m^2}{4E}
\begin{pmatrix}
- 2 \cos 2 \theta & \sin 2 \theta \\
\sin 2 \theta & 0 \\
\end{pmatrix}.
\end{equation}

where ${\theta}$ is the mixing angle associated with the 2x2 PMNS matrix.
If this effect leaves the neutrino momentum unchanged, the resulting additional terms in the Hamiltonian may interfere with the propagation of the unscattered neutrinos.
The potential includes contributions from both charged current and neutral current interactions, although the charged current interactions arise solely from the electron neutrinos.
The potential, expressed in the flavor basis, is then

\begin{equation}
V_{CC,\alpha} = 
\begin{cases}
\sqrt{2} \pm G_F n_e \left( x\right) & \alpha = e \\
0 & \alpha = \mu, \tau
\end{cases}\ \ \ \ \ 
V_{NC,\alpha} = -\frac{G_F}{\sqrt{2}} n_e\left(x\right)\;\;\; \alpha = e, \mu, \tau
\end{equation}

where a + is used for neutrinos and a - is used for antineutrinos, ${n_e}$ is the density of electrons in the medium, and ${G_F}$ is the Fermi coupling constant.
Note that the angle included here is that of the PMNS matrix in two dimensions.
A full description of three flavor neutrino oscillation in the presence of a matter potential is beyond the scope of this work.
Further information and explicit forms may be found in \cite{Review-PMNS,Review-PMNS2}.
The full three-flavor oscillation calculation is used for this thesis using the Prob3++ code \cite{Barger-Oscillations,prob3}, which includes an implementation of matter effects.
The electron densities are calculated from the Preliminary Reference Earth Model (PREM) \cite{PREM}.
