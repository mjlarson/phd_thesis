\section{The Properties of the GRECO Event Selection}
The completion of cuts yields the completed GRECO event selection.
There are many ways to characterize the final event sample.

One of the most general methods is via the \textbf{effective area}, a theoretical construct designed to allow a generic flux to be propagated to the selection.
The effective area is a representation of the approximate cross section of a theoretical ideal detector.
By combining the effective area and an arbitrary flux, an event rate can be obtained.

The ratio of the effective area at generation level and at a later cut level gives the efficiency of the selection.
Because the event selection efficiency can change as a function of energy and direction, 

The GRECO effective area for each of the neutrino channels is shown in \ref{fig:effective_areas}.




\label{subsubsec:greco_truth}
\subsection{Energy and Zenith Reach}


