\section{Simulation Efficiency with KDE Filtering}
The production of background atmospheric muons is the most computationally expensive part of oscillation analyses in IceCube. 
For the primary analysis of this thesis, full CORSIKA simulations have proven to be impractical. 
When simulating at the detector threshold, many of the produced showers lead to no significant contribution to the detector output, leading to significant inefficiency.
In addition, the choice of cuts employed (see \ref{chapter:greco}) lead to a low simulation efficiency at final level that has both strong energy and zenith dependences.

For this work, a set of $3\times 10^{9}$ muons were produced using collaboration resources over a period of approximately three months.
Muons were produced following the MuonGun scheme described in \ref{subsubsec:muongun}, with all muons aimed toward the DeepCore fiducial volume.
The resulting generation-level distributions are shown in \ref{fig:mg_generation}.
The muons follow the offset power law described in \ref{subsubsec:muongun} in energy and the expected zenith angle spectrum from the underlying flux model of \cite{h4a}.
The generated muons show azimuthal symmetry as expected.

After processing to the final level of the GRECO event selection (see \ref{chapter:greco}, the background muon simulation retains approximately 9000 simulated events of the original sample.
This remaining subsample may be used to estimate the simulation efficiency of muons in this selection.
The efficiency is most easily observed with low energy muons that travel very downgoing, as seen in \ref{fig:mg_ineff}.
These events make up the majority of simulated muons at generation level, but no such events reach the final level of the analysis.

In addition to the energy- and zenith-dependent effects, the GRECO selection exhibits strong azimuthal selection bias.
This arises due to two main effects. 
The first is the offset between the center of IceCube (and therefore the center of the generation volume) at $(x,y,z)=(0,0,0)$ and DeepCore at $(x,y,z)=(XXXXXXX)$.
This gives rise to an azimuthal effect appearing as a sinusoidal variation of the minimum generated energy of events at final level.

The second is the regular hexagonal structure of the IceCube volume, with long "corridors" through which muons may reach DeepCore without crossing any strings.
Cuts designed to look for hits left in the veto region produce these azimuthal biases when muons close to strings (ie, outside of the "corridors") are more likely to be identified and removed than those further from strings (inside of the "corridors").

In order to improve simulation statistics at final level, the existing MuonGun simulation scheme was modified to include an energy-, zenith-, and azimuthally-dependent prescale factor.
This approach, here entitled \textbf{KDE Filtering}, allows simulation to be produced with a known bias matching that of a given set of input files.

In this scheme, the \textbf{kernal density estimator} (\textbf{KDE}) from SciPy \cite{scipy} is applied to all remaining events at final level of the GRECO event selection. 
The KDE uses a Gaussian kernal to represent each event in energy, zenith, and azimuth. 
The resulting KDE is normalized to produce an approximation of the final selection probability density function.

In the new simulation scheme, an event is produced using standard settings for MuonGun generation. 
Immediately following generation, the probability of the new event reaching final level is calculated from the KDE, with typical values 
of approximately $10^{-4}$ for a likely event and $10^{-9}$ or lower for unlikely events.
A prescale multiplicative factor of $10^5$ is used to set the overall probability scale.
The product, $p$ of the prescale factor and KDE-derived probability may not exceed 100\% and any values for which this may be the case are directly set to 100\%.

Using a random number generator, this $p$ factor is used to retain or reject the muon event.
The simulation then proceeds as normal, with photon propagation, detector simulation, triggering, and filtering.

When events need to be weighted by a flux model, the $p$ factor must be included as well. 
The modification to the weighting scheme requires the use of the original MuonGun weighting code in the usual scheme. 
The total weight is then scaled by $\frac{1}/{p}$, which corrects the rates and uncertainties for the simulated events.

By removing unlikely events early in the simulation chain, the required computational resources are significantly reduced.
In addition, the updated simulation scheme gives the opportunity, for the first time, to study various detector systematics affecting the atmospheric muons at the final level of oscillation analyses.

By way of comparison, the total amount of time required to produce 9000 events at final level of the GRECO selection with similar available resources reduces from three months under the standard scheme to approximately four days using the KDE filtering method.

