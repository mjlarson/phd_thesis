\label{sec:likelihoods}
\section{The Method of Maximum Likelihood}

\label{subsec:chi2}
\subsection{The $\chi^2$ Fit}
The simplest implementation of a fitting algorithm begins with an assumption of the true and observed distributions.
Namely, that the observed number of events in each bin of the histogram is drawn from a distribution approximately Gaussian with a mean $\mu$ equal to the expectation from simulations and a variance $\sigma^2$ calculated from the Poisson uncertainty on the expectation. 

\begin{equation}
	P\left(x|\mu\right) = N e^{\frac{1}{2}\frac{\left(x-\mu\right)^2}{\sigma^2}}
\end{equation}

where $N$ is a normalization constant and, in the case of Poissonian statistics of simple histograms, the variance is given by the event weights in the specified bin.

\begin{equation}
	\sigma^2 = \mu = \sum{w}
\end{equation}

\improvement{This needs work. can't even be called a derivation. its just crap.}
From this point, taking the logrithm yields the standard $\chi^2$ definition for the likelihood after dropping the constant terms.

\begin{equation}
	\chi^2 = \frac{\left(x-\mu\right)^2}{\mu}
\end{equation}	

\label{subsec:finite_stats}
\subsection{Finite Statistics}
The $\chi^2$ distribution above implicitly assumes that the dominant source of uncertainty at the best-fit point comes from the statistical fluctuations of the data around the true distribution represented by the Monte Carlo simulation.
While this is true in te ideal case, in practice the statistical properties of the simulation sets themselves cannot be ignored.
In general, every attempt should be made to ensure the statistical fluctuations of the simulation sets are negligible compared to those of the data.
This typically leads to requests for at least an order of magnitude larger simulation statistics than expected from the data itself.
\needfig{make a plot showing chi2 value as a function of mc stats scale factor to justify the 10x rule}
In the situation where this is infeasible, modifications to the likelihood space itself may be used to account for the additional uncertainties.
For this analysis the statistical uncertainties of the underlying simulation sets are added to the weighted uncertainties in quadrature.

\label{eqn:mchi2}
\begin{equation}
	\chi^2_{1} =\frac{1}{2}\frac{\left(x-\sum w\right)^2}{\left(\sum{w}\right)^2 + \sum{w^2}}
\end{equation}		

Due to the large uncertainties associated with the atmospheric muon sample, further considerations are necessary.
In particular, the large uncertainties associated with atmospheric muon simulation statistics may be used by the fitter in order to reduce the $\chi^2_{FS}$ value.
This situation proceeds with the minimization process as normal until a runaway effect is observed by increasing the statistical uncertainties at the expense of data/simulation agreement.
In this case, the numerator becomes simply

\label{eqn:mchi2_num}
\begin{equation}
	\lim_{N_{\mu}\rightarrow\infty} \left(x-\sum w\right)^2 = \left(\sum w\right)^2
\end{equation}

The resulting limit in each bin as the event weights become large is therefore

\label{eqn:mchi2_lim}
\begin{equation}
	\lim_{N_{\mu}\rightarrow\infty} \chi^2_{1} =  \frac{\left(\sum w\right)^2}{\left(\sum{w}\right)^2 + \sum{w^2}}
\end{equation}
\begin{equation}
	\lim_{N_{\mu}\rightarrow\infty} \chi^2_{1} = 0
\end{equation}

While this is a particular concern for all simulation types, the dominant contribution to the $\sum{w^2}$ term is the atmospheric muons. 
In addition, the atmospheric muons have the strongest impacts from non-normalization systematic uncertaintines, particularly the DOM efficiency and the absorption.
Modifying either of these parameters or the normalization systematics in the fit may lead to this runaway behavior.

In order to prevent this situation, a further modification of the $\chi^2$ is necessary.
For this analysis, the total scale of the statistical uncertainty is assumed to be set by the seed values of the fit.

\label{eqn:w2_constant}
\begin{equation}
	N_{w^2} = \frac{\sum{w^2_{Seed}}}{\sum{w^2}}
\end{equation}

With this modification, the $\chi^2$ is now defined to be \improvement{this is only true for muons! how do i explain that?}

\label{eqn:chi2_final}
\begin{equation}
	\chi^2_{FS} =\frac{1}{2}\frac{\left(x-\sum w\right)^2}{\left(\sum{w}\right)^2 + N_{w^2}\sum{w^2}}
\end{equation}

Taking the limit of this $\chi^2$ in a single bin as the event weights become large now gives

\begin{equation}
	\lim_{N_{\mu}\rightarrow\infty}  \chi^2_{FS} =\frac{1}{2}\frac{\left(x-\sum w\right)^2}{\left(\sum{w}\right)^2 + N_{w^2}\sum{w^2}}
\end{equation}
\begin{equation}
	\lim_{N_{\mu}\rightarrow\infty} \chi^2_{FS} = 1
\end{equation}

This is in agreement with the limit obtained from the standard $\chi^2$.

\subsection{Fit Priors}
In many cases, the systematics listed in \ref{sec:systematics} have known constraints from external measurements.
This information can be useful and should be included in the analysis to bias the minimization toward the most likely systematics values.
These constraints are included in the form of \emph{priors} in the analysis.
Priors are additional terms included multiplicatively (additively) in the likelihood (log-likelihood) calculation.
These often take the form of a Gaussian distribution with mean $\mu$ and variance $\sigma^2$ given by external measurements.
For this measurement, most priors are handled assuming a standard Gaussian form.
For a systematic $i$ with value $x_i$, these additional terms take the form

\begin{equation}
	\chi^2_{i} =  \frac{\left(x_i-x_0\right)^2}{\sigma^2}
\end{equation}

These additional terms are added to \ref{eqn:chi2_final} in order to calculate the final $\chi^2_{FS}$ used in the minimization for this analysis.

\begin{equation}
	\chi^2_{Total} = \sum_{bins} \chi^2_{FS} + \sum_{systematics} \chi^2_{i}
\end{equation}

A list of priors is shown in \ref{tab:priors}.\findref{All of these references for the priors...}
Note that the coincident event fraction is effectively a one-sided Gaussian due to physical constraints on the value.


% Please add the following required packages to your document preamble:
\begin{landscape}
\begin{table}[]
\centering
\begin{tabular}{@{}llllllll@{}}
\toprule
                               & Systematic                            & Unit                        & Type       & Baseline/Seed Value     & Prior           & Allowed Range & Reference                     \\ \midrule
Physics Parameter    & $N_{\nu_\tau}$                    & -                             & Analytic   & 1.0                     & -                       & 0.0 - 2.0     & -                             \\ \midrule
\multirow{2}{*}{Oscillations}  & $\Delta m^2_{3j}$  & $10^{-3}$ $eV^2$ & Analytic   & 2.526                  & -                      & 2.0 - 3.0     & nufit             \\
                               & Sin$^2 \theta_{23}$            & -                             & Analytic   & 0.440 (NO), 0.66 (IO)   & -              & 0.0 - 1.0     & nufit             \\ \midrule
Total Rates              & $N_nu$, $N_mu$                  & Years                      & Analytic   & 2.25                    & -                      & 0.0 - 10.0    & -                             \\ \midrule
\multirow{3}{*}{Cross-section} & Axial Mass (QE)     & $\sigma$                & Analytic   & 0.0                     & 0.0 $\pm$ 1.0   & -5.0 - 5.0    & GENIE             \\
                               & Axial Mass (RES)                    & $\sigma$               & Analytic   & 0.0                     & 0.0 $\pm$ 1.0   & -5.0 - 5.0    & GENIE             \\
                               & $N_{\nu^{NC}}$                   & -                            & Analytic   & 1.0                     & 1.0 $\pm$ 0.2   & 0.0 - 2.0     & NC prior          \\ \midrule
\multirow{6}{*}{Flux}& $\gamma_\nu$                    & -                            & Analytic   & 0.0                     & 0.0 $\pm$ 0.10  & -0.50 - 0.50  & Honda             \\
                               & $\gamma_\mu$                   & $\sigma$                & Analytic   & 0.0                     & 0.0 $\pm$ 1.0   & -5.0 - 5.0    & ste and davide's paper       \\
                               & Up/Horizontal Ratio               & $\sigma$                & Analytic   & 0.0                     & 0.0 $\pm$ 1.0   & -5.0 - 5.0    & Barr              \\
                               & $\nu$/$\bar{\nu}$ Ratio       & $\sigma$               & Analytic   & 0.0                     & 0.0 $\pm$ 1.0   & -5.0 - 5.0    & Barr              \\
                               & $\Phi_{\nu_e}$                    & -                             & Analytic   & 1.0                     & 1.0 $\pm$ 0.05 & 0.8 - 1.2     & Barr              \\
                               & Coincident Fraction                & -                            & Hyperplane & 0.0                     & 0.0 + 0.10        & 0.0 - 1.0     & coin fraction...? \\ \midrule
\multirow{5}{*}{Detector}      & DOM Efficiency        & -                             & Hyperplane & 1.0                     & 1.0 $\pm$ 0.1  & 0.7 - 1.3     & DOM eff           \\
                               & Hole Ice (Lateral)                   & -                            & Hyperplane & 0.25                    & 0.25 $\pm$ 0.10 & 0.0 - 0.5  & hole ice          \\
                               & Hole Ice (Forward)                 & -                            & Hyperplane & 0.0                     & -                        & -5.0 - 5.0    & hole ice          \\
                               & Absorption                            & -                            & Hyperplane & 1.0                     & 1.0 $\pm$ 0.1   & 0.5 - 1.5     & bulk ice          \\
                               & Scattering                             & -                            & Hyperplane & 1.0                     & 1.0 $\pm$ 0.1   & 0.5 - 1.5     & bulk ice          \\ \bottomrule
\end{tabular}
\caption{Priors and allowed ranges for each systematic included in this analysis.}
\label{tab:priors}
\end{table}
\end{landscape}
