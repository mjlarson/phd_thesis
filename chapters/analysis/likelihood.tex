\label{sec:likelihoods}
\section{The Method of Maximum Likelihood}

\label{subsec:chi2}
\subsection{The $\chi^2$ Fit}
The simplest implementation of a fitting algorithm begins with an assumption of the true and observed distributions.
Namely, that the observed number of events in each bin of the histogram is drawn from a distribution approximately Gaussian with a mean $\mu$ equal to the expectation from simulations and a variance $\sigma^2$ calculated from the Poisson uncertainty on the expectation. 

\begin{equation}
	P\left(x|\mu\right) = N e^{\frac{1}{2}\frac{\left(x-\mu\right)^2}{\sigma^2}}
\end{equation}

where $N$ is a normalization constant and, in the case of Poissonian statistics of simple histograms, the variance is given by the event weights in the specified bin.

\begin{equation}
	\sigma^2 = \mu = \sum{w}
\end{equation}

From this point, taking the logrithm yields the standard $\chi^2$ definition for the likelihood after dropping the constant terms.

\begin{equation}
	\chi^2 = \frac{\left(x-\mu\right)^2}{\mu}
\end{equation}	



\label{subsec:finite_stats}
\subsection{Finite Statistics}
The $\chi^2$ distribution above implicitly assumes that the dominant source of uncertainty at the best-fit point comes from the statistical fluctuations of the data around the true distribution represented by the Monte Carlo simulation.
While this is true in te ideal case, in practice the statistical properties of the simulation sets themselves cannot be ignored.
In general, every attempt should be made to ensure the statistical fluctuations of the simulation sets are negligible compared to those of the data.
This typically leads to requests for at least an order of magnitude larger simulation statistics than expected from the data itself.
\needfig{make a plot showing chi2 value as a function of mc stats scale factor to justify the 10x rule}
In the situation where this is infeasible, modifications to the likelihood space itself may be used to account for the additional uncertainties.
For this analysis the statistical uncertainties of the underlying simulation sets are added to the weighted uncertainties in quadrature.

\label{eqn:mchi2}
\begin{equation}
	\chi^2_{1} =\frac{1}{2}\frac{\left(x-\sum w\right)^2}{\left(\sum{w}\right)^2 + \sum{w^2}}
\end{equation}		

Due to the large uncertainties associated with the atmospheric muon sample, further considerations are necessary.
In particular, the large uncertainties associated with atmospheric muon simulation statistics may be used by the fitter in order to reduce the $\chi^2_{FS}$ value.
This situation proceeds with the minimization process as normal until a runaway effect is observed by increasing the statistical uncertainties at the expense of data/simulation agreement.
In this case, the numerator becomes simply

\label{eqn:mchi2_num}
\begin{equation}
	\lim_{N_{\mu}\rightarrow\infty} \left(x-\sum w\right)^2 = \left(\sum w\right)^2
\end{equation}

The resulting limit as the event weights become large is therefore

\label{eqn:mchi2_lim}
\begin{equation}
	\lim_{N_{\mu}\rightarrow\infty} \chi^2_{1} =  \frac{\left(\sum w\right)^2}{\left(\sum{w}\right)^2 + \sum{w^2}}
\end{equation}
\begin{equation}
	\lim_{N_{\mu}\rightarrow\infty} \chi^2_{1} = 0
\end{equation}


While this is a particular concern for all simulation types, the dominant contribution to the $sum{w^2}$ term is the atmospheric muons. 
In addition, the atmospheric muons have the strongest impacts from non-normalization systematic uncertaintines, particularly the DOM efficiency and the absorption.
Modifying either of these parameters or the normalization systematics in the fit may lead to this runaway behavior.

In order to prevent this situation, a further modification of the $\chi^2$ is necessary.
For this analysis, the total scale of the statistical uncertainty is assumed to be set by the seed values of the fit.

\label{eqn:w2_constant}
\begin{equation}
	N_{w^2} = \frac{\sum{w^2_{Seed}}}{\sum{w^2}}
\end{equation}

With this modification, the $\chi^2$ is now defined to be

\begin{equation}
	\chi^2_{FS} =\frac{1}{2}\frac{\left(x-\sum w\right)^2}{\left(\sum{w}\right)^2 + N_{w^2}\sum{w^2}}
\end{equation}