\section{Expectations from Monte Carlo}

\label{subsec:binning}
\subsection{Choice of Binning}
In order to understand the potential for IceCube's measurement of $\nu_\tau$ appearance, a choice of binning must be decided upon.
The analysis discussed here uses two variables to describe the oscillations: the reconstructed energy and zenith angle.
These dimensions form an integral part of the standard oscillation analysis and are often used in measurements of atmospheric mixing parameters \findref{list of atmo disappearance measurements that use zenith and energy binning}. 

The choice of binning for zenith angles is selected to be similar, but somewhat finer than previous work \findref{dragon, leesard 3 year papers}.
For this work, we use the fully sky, including upgoing events ($\mathtt{cos(\phi)}=-1$) corresponding to events passing through the full diameter of the Earth where we expect the strongest oscillation effects to very downgoing events ($\mathtt{cos(\phi)}=1$) where events are originating in showers above the Antarctic.
The energy binning is selected to match previous work from DeepCore and consists of 8 bins logarithmically spaced from 5.6 GeV to 56 GeV \findref{dragon, leesard}.

In addition, recent work with DeepCore has shown that a third dimension separating the sample into cascade-like and track-like events may provide better sensitivity than using solely track-like events.
Two variables are available in the GRECO sample.
The first, the reconstructed length of a muon track, provides a simple separation between events with a clear muon track from those without one.
This, in general, leads to reasonable separation between the $\nu_\mu$ events undergoing disappearance and $\nu_\tau$ events undergoing appearance.
This may be seen in \needfig{cumulative plot of track length}, where the cumulative distribution of the various simulation components are shown as a function of the reconstructed track length.
The separation between the $\nu_\mu$ and $\nu_\tau$ charged-current samples occurs between 30-50 meters.
By separating the sample into cascade-like events (eg. L < 50 m) and track-like events (L $\geq$ 50 m), the disappearance and appearance may be partially disentangled.

The second potential separating variable is the likelihood ratio between a cascade and PegLeg's mixed cascade+track reconstructions.
A higher likelihood (lower log-likelihood) in the casade fit implies that the event is more likely intrinsically cascade-like while the reverse is true for intrinsically track-like events.
The cumulative plot of the likelihood ratio is shown in \needfig{cumulative plot of deltallh}. 
There exists a broad choice of values with similar separation properties from approximuately $-4 < \mathtt{\Delta LLH} < -2$.
Once again, separating events into two samples using the likelihood ratio may improve the ability of the analysis to disentangle the disappearance and appearance effects.

Both variables clearly show some separating power and likely have similar behavior: an event with a longer reconstructed muon track should be expected to prefer the PegLeg reconstruction over a cascade reconstruction.
In order to choose between the parameters, the efficacy of separating each of the simuluation samples from the $\nu_\tau$ charged current signal was evaluated.
The results are shown in \needfig{roc curves for track length}, which give the fraction of each type rejected and the number of $\nu_\tau$ events accepted into the "cascade-like" sample for various choices of the separating parameters. \improvement{this sentence needs to be reworded. its too verbose}
Values further from a diagonal indicate better separation between the $\nu_\tau$ and other event types.
Here we see that the track length performs uniformly better than the likelihood ratio in separating the disappearing $\nu_\mu$ charged-current and appearing $\nu_\tau$ charged current events.
Furthermore, the reconstructed track length performs significantly better in separating the neutrino components from the atmospheric muon background.
The reconstructed track length is therefore selected as the separating variable for this analysis.

\label{subsec:fit_templates}
\subsection{The MC Fit Templates}
A choice of 50 meters of reconstructed track length is selected for this analysis. 
Because the PegLeg reconstruction assumes the muon track to be minimally ionizing, the division of track-like and cascade-like has an effect on the minimum energy of each sample.
In particular, no track-like events (L $\geq$ 50 m) may have less than 10 GeV in total reconstructed energy.
Both track- and cascade-like events may reconstruct with higher energies than 10 GeV.

The binned expectations used in the fit are shown in \needfig{mc templates!} assuming the oscillation parameters given by \findref{nufit 2.2}.
The lack of expected events in the track-like histogram is clearly visible.
As expected, atmospheric muon background events tend to reconstruct as downgoing events, primarily visible in the track-like channel.
The signal $\nu_\tau$ events occur in the very upgoing cascade channel and make up, at most, approximately 10\% of the events in those bins.

