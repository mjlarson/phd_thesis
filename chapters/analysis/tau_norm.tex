\section{Parametrizing the Tau Neutrino Appearance}
In order to properly measure the appearance of $\nu_\tau$ events, a choise of "appearance parameter" must be selected.
Here, we discuss the choice of parameter used in this analysis.

\label{subsec:cc_vs_ccnc}
\subsection{CC vs CC+NC}
As described in \ref{sec:detection_methods}, neutrinos may interact in two distinct ways to produce light in the IceCube detector.
These two methods, the charged-current and neutral-current interactions, provide separate windows into neutrino interactions.
Tau neutrino events may interact in either of these channels depending on the neutrino energy.

With a mass of 1776.82 $\pm$ 0.16 MeV and a lifetime of 290.3 $\pm$ 0.5 femtoseconds \findref{PDG}, $\tau$ leptons produced during neutrino oscillations in DeepCore tend to travel very short differences before decaying.
The charged-current interactions of the $\nu_\tau$ result in a variety of signatures due to the unique decay behavior of the $\tau$ lepton.

\label{eqn:tau_decay_modes}
\begin{equation}
	\tau^- \rightarrow 
		\begin{cases} 
			\mu^- \bar{\nu}_\mu \nu_\tau & \mbox{17.41 $\pm$ 0.04\%} \\ 
			e^- \bar{\nu}_e \nu_\tau & \mbox{17.83 $\pm$ 0.04\%} \\ 
			\mbox{Hadrons} & \mbox{Otherwise} \\ 
		\end{cases}
\end{equation}

In either the muonic or the electronic decay modes, a fraction of the energy is lost to outgoing neutrinos, resulting in a smaller observed charge than would be associated with a corresponding interaction of another neutrino type.
Furthermore, the muonic decay mode may lead to a visible muon track for the $\nu_\tau$ interaction.
These muon tracks associated with the appearance of $\nu_\tau$ would appear at lower energies than the tracks corresponding to the $\nu_\mu$ disappearance, allowing both effects to be observed simultaneously.

Unlike the varied results of the charged current interactions, neutral current interactions of neutrinos are assumed to have identical coupling and behavior, regardless of flavor and, therefore, undergo no observable change due to oscillations.
Because of this, studies of the standard unitary PMNS matrix tend to treat neutral current events as effectively non-oscillating \findref{superk paper, opera paper sources for unoscillating NC}.
In contrast, searches for new physics and sterile neutrinos result can result in a change in the apparent number of neutral current interactions in the detector.

For this analysis, both approaches have been adopted.
A fit using charged-current events as the signal is used to provide limits on the modifications to a 3x3 mixing matrix without the introduction of neutral-current altering behavior \findref{non-sterile explanations of non-unitarity? maybe the neutrino decay paper?}.
A second fit, including both neutral current and charged current $\nu_\tau$ events, provides more insight into possible extra flavors of neutrinos.

\label{subsec:norm_tau}
\subsection{The $\nu_{\tau}$ Normalization}
Because effectively all $\nu_\tau$ events observable in DeepCore are the result of neutrino oscillations, the total number of observed $\nu_\tau$ interactions is a direct measure of the appearance itself.
The number of $\nu_\tau$ events interacting in DeepCore is, however, affected by many of the previously-discussed systematics.
In particular, the number of events is strongly related to the assumed atmospheric oscillation parameters.

In order to provide a quantitative measure of the appearance, the overall normalization of signal events is used as a final physics parameter. \improvement{think up a better phrasing to introduce the tau normalization}
The normalization is a fit parameter, defined to be a total modification of the number of candidate $\nu_\tau$ events after all other systematic parameters are applied.

\label{eqn:norm_tau_definition}
\begin{equation}
	f'_{ijk} = \sum_{m\neq\nu_\tau} f^m_{ijk}\left(\theta_{23}, \Delta m^2, ...\right) + N_{\nu_\tau} f^{\nu_\tau}_{ijk}\left(\theta_{23}, \Delta m^2, ...\right) 
\end{equation}

In this case, we end up with two general cases for the result.
In the expected case, $N_{\nu_\tau}=1.0$, we find that the number of candidate events is consistent with our modeling of the $\nu_\tau$ and unitary PMNS mixing.
If the value is significantly different from 1.0, we may have hints of either mismodeled cross-sections or of novel physics. \findref{Crazy shit that I will probably take out. but maybe find the neutrino decay paper again?}
Due to the large existing uncertainties in the PMNS matrix described in \ref{sec:current_limits}, either situation is likely to yield valuable information.


\label{subsec:superk_and_opera}
\subsection{Limits on the $\nu_\tau$ Normalization}
This analysis is not the first to search for appearance in this way.
Two other experiments, OPERA and Super-Kamiokande, have reported previous measurements parametrized in the same way.

\label{subsubsec:opera_limit}
\subsubsection{The OPERA Limit}
The Oscillation Project with Emulsion-tRacking Apparatus, better known by the acronym OPERA, is an experiment designed to search for  $\nu_\tau$ appearance.
Unlike IceCube's use of atmospheric neutrinos, OPERA uses muon neutrinos produced in the CERN Neutrinos to Gran Sasso (CNGS) beamline.
OPERA uses an bricks of photographic films in order to accurately track and reconstruct neutrino interactions in the fiducial volume.
This technique allows analysizers to clearly identify not only the initial neutrino interactionv vertex, but also the decay products along the path of the charged lepton produced in charged current interactions.
In OPERA, the muon and tau lepton produce significantly different signals due to the short lifetime and unique decay properties of the tau lepton.
The impressive ability to identify the particle dynamics is balanced by the small fiducial volume of the experiment, yielding only 5408 useful events for analysis from five years of data-taking.

In 2015, OPERA released their final result in the search for $\nu_\tau$ appearance.
Five candidate events, shown in \needfig{opera tau neutrino event views},  were identified in the data sample with a signal expectation of 2.64 $\pm$ 0.53 and a background expectation of 0.25 $\pm$ 0.05.
The data unambiguously rules out the no-appearance hypothesis, with a rejection at 5.1$\sigma$ \findref{opera paper: https://arxiv.org/abs/1507.01417}.

In terms of the normalization described above, OPERA reported a final value of $1.8^{+1.8}_{-1.1}$ at the 90\% level. 
This value is consistant with the standard unitary oscillation scheme, but with large errors.

\label{subsubsec:superk_limit}
\subsubsection{The Super-Kamiokande Limit}
Super-Kamiokande, described in \ref{sec:superk_atmo}, also has reported results in searches for $\nu_\tau$ appearance.
The Super-Kamiokande collaboration developed a new event selection in the search for $\nu_\tau$ events, including the implementation of a neural net \findref{superk paper on appearance https://arxiv.org/pdf/1711.09436.pdf} to identify $\tau$-like and non-$\tau$-like events.
The neural net itself includes information about the energy of the event and is trained against a background sample of simulated events.
Events are analyzed in terms of the zenith angle and the neural net output variable.

These two categories of events are fit with an unbinned likelihood including 28 systematic effects included in the analysis.

\begin{table}[]
\centering
\begin{tabular}{@{}llll@{}}
\toprule
Interaction Mode & Non-tau-like & tau-like & All    \\ \midrule
CC nue       & 3071.0          & 1399.2      & 4470.2 \\
CC numu     & 4231.9          & 783.4       & 5015.3 \\
CC nutau    & 49.1            & 136.1       & 185.2  \\
NC               & 291.8           & 548.3       & 840.1  \\ \bottomrule
\end{tabular}
\caption{The rates expected for each of the neutrino types in the Super-Kamiokande search for $\nu_\tau$ appearance. Reproduced from.}
\label{tab:superk_appearance_rates}
\end{table}
 \findref{https://arxiv.org/pdf/1711.09436.pdf again}{here}

The Super-Kamiokande measurement yields an expectation of 185.2 $\nu_\tau$ events in 5326 days or appoximately 12.7 events per year \unsure{wtf is going on here? table 1 in the paper gives an expectation of 185.2 events, but the stuff at the top right of page 11 says the expectation is 224...?! and NIETHER of these give the 1.47 that they quote. wtf}. 
After fitting, the final rejection of the no-appearance hypothesis is found to be 4.6$\sigma$.
Like OPERA, Super-Kamiokande finds more $\nu_\tau$ candidate events than expected, with a best-fit normalization of $1.47 \pm 0.32$.

\needfig{figure 14 of https://arxiv.org/pdf/1711.09436.pdf}
