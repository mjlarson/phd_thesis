%%%%%%%%%%%%%%%%%%%%%%%%%%%%%%%
% Submitted in fullfullment stuff
%%%%%%%%%%%%%%%%%%%%%%%%%%%%%%%
\renewcommand{\abstractname}{Fullfillment}
\thispagestyle{plain}
\begin{abstract}
\pagenumbering{roman}
\setcounter{page}{2}

\begin{center}
Submitted in fullfilment of the requirements\linebreak
for the degree of \linebreak
Doctorate in Philosophy\linebreak
at \linebreak
the Niels Bohr Institute, University of Copenhagen.
\end{center}

\end{abstract}

\let\oldcleardoublepage\cleardoublepage % Save the regular one
\let\cleardoublepage\clearpage % Define a new one

%%%%%%%%%%%%%%%%%%%%%%%%%%%%%%%
% Normal abstract
%%%%%%%%%%%%%%%%%%%%%%%%%%%%%%%
\renewcommand{\abstractname}{Abstract}
\begin{abstract}
\thispagestyle{plain}

\begin{center} \textbf{Abstract} \end{center}

Neutrino have been known to oscillate between the three flavors since the first discoveries two decades ago.
Over that time, our knowledge of the parameters which govern these oscillations has improved significantly.
The largest remaining uncertainties in the measurement of neutrino oscillations are those that govern the tau neutrino.
In this thesis, a direct measurement of tau neutrino oscillations is performed with the IceCube Neutrino Observatory located in the ice deep beneath the South Pole.

The measurement of atmospheric tau neutrino appearance requires a precise understanding of backgrounds.
In order to perform the measurement, improvements to the modeling of the detector noise have been performed, reducing the uncertainties in the noise model used in IceCube significantly.
Additional improvements to the simulation efficiency investigated during this thesis reduce the computational requirements of atmospheric muon background events by more than three orders of magnitude.
These improvements allow, for the first time, the use of simulation of background events in oscillation measurements performed by IceCube.

Using the DeepCore detector, a densely instrumented infill of IceCube located in the clearest ice of the Antarctic glacier, a new selection of events has been created in the search for tau neutrino appearance from atmospheric oscillations.
Tau neutrino appearance and muon neutrino disappearance were measured simultaneously with the new sample from 5.6 to 56~GeV from data collected over a period of 968 days.
The best fit values, $N^{CC}_\tau=0.566^{+0.356}_{-0.303}$ for the charged current exclusive channel and $N^{NC+CC}_\tau=0.733^{+0.305}_{-0.243}$ for the neutral current inclusive channel, improve upon previous measurements set by other experiments.

\end{abstract}

\let\cleardoublepage\oldcleardoublepage % Return to the old one
\cleardoublepage

%%%%%%%%%%%%%%%%%%%%%%%%%%%%%%%
% Danish abstract
%%%%%%%%%%%%%%%%%%%%%%%%%%%%%%%
\renewcommand{\abstractname}{Resume}
\begin{abstract}
\thispagestyle{plain}

\begin{center} \textbf{Resum\'{e}} \end{center}

Siden de første opdagelser af fænomenet for tyve år siden, har vi vidst at neutrinoer oscillerer mellem deres tre familier. Siden da er vores viden om neutrinoernes oscillations parametre vokset betragteligt. De største tilbageværende usikkerheder i målingen af neutrino oscillationer, vedrører tau-neutrinoen. I denne afhandling præsenteres resultaterne af en direkte måling af oscillationen af atmosfæriske neutrinoer til tau-neutrinoer, udført med IceCube Neutrino Observatoriet, beliggende dybt i isen ved Sydpolen.

Målingen af atmosfæriske tau-neutrinoer kræver overordentligt godt kendskab til de systematiske baggrundssignaler. For at kunne genneføre målingen af tau-neutrinoer, udvikledes en forbedret model for detektorens støj niveau, hvilket resulterede i en kraftig reduktion i usikkerheden for denne baggrund i IceCube.  Yderligere forbedringer i effektiviteten af simuleringen af den atmosfæriske muon baggrund reducerede kravende til computerresourcer med mere end tre størrelsesordner. Disse forbedringer tillader, for første gang, brugen af simulerede baggrundsbegivenheder i en oscillationsmåling med IceCube.

Ved at bruge DeepCore detektoren, en tæt instrumenteret del af IceCube placeret i den klareste is i gletcheren ved Sydpolen, kunne en ny klasse af begivenheder udvælges, med det særlige mål for øje at måle tilsynekomsten af atmosfæriske tau-neutrinoer via neutrino oscillationer. Både andelen af atmosfæriske neutrinoer der oscillere til tau-neutrinoer og muon-neutrinoer der oscillere til en anden af de tre familier, er blevet målt i energiområdet 5.6 til 56 GeV med denne nye udvælgelse af data, indsamlet over 968 dage.

Det bedste fit til data giver at $N^{CC}_\tau=0.566^{+0.356}_{-0.303}$ når der kun kigges på charged current interaction kanalen og $N^{NC+CC}_\tau=0.733^{+0.305}_{-0.243}$ når der kigges på kanalen der også inkluderer neutral current begivenheder. Begge værdier er bedremålinger end andre eksperimenter tidligere har kunne offentliggøre.

\end{abstract}

%%%%%%%%%%%%%%%%%%%%%%%%%%%%%%%
% acknowledgements
%%%%%%%%%%%%%%%%%%%%%%%%%%%%%%%
\renewcommand{\abstractname}{Acknowledgements}
\begin{abstract}
\thispagestyle{plain}

\begin{center} \textbf{Acknowledgements} \end{center}

I have now spent a total of nine years of my life working in IceCube.
Over that time, I've had a number of amazing opportunities and met some incredible people.

I'd like to thank Jason (and the rest of the Koskinens) without whom I would quite literally not be here. 
You've been the closest I've had to a family here for years and I can't think you enough for the kindness and patience you've shown me.
Thank you for letting me be a part of your lives for these last nine years.

And to you in particular, Jason: somehow you've managed to put up with me all of these years and haven't yet killed me. 
I can only imagine the patience you've had to master as I've fumbled my way through one deadline after another on the road through this PhD.
Thanks for always being around and willing to help me through it all.
I hope you feel as happy as I do to finally get this analysis over with.

The people in my office over the years have come and gone, but I'll always remember the general tone of barely concealed chaos.
I think that may have been mostly Joakim and Eva, although maybe Morten also got a little bit of that action sometimes.
I think I'll always have great memories of sitting in the office making beers.

IceCube people not at NBI: you've been an incredible bunch over the years and I look forward to working together again in the future!
That goes doubly for Samuel, Lisa, Elim, Martin, Sarah, and Melanie. 
You five have been my closest friends in the collaboration over the last few years and I can't wait to see what we can do together once we don't have theses to worry about any more!

And finally: my actual family.
I think my mother would kill me if I didn't mention her somewhere, so here goes: Hi Mom! Hi Dad! 
Hope you're happy that my stress and white hairs might actually let up now!
I mean, maybe. Possibly. Probably shouldn't get my hopes up.

Okay, they're not reading anymore. 
Almut: you have been the greatest thing to happen to me since I moved to Copenhagen. 
I cannot begin to explain to you how much I appreciate your support through the last few years.
You've been around for every failure, every frustration, and every victory I've had and you're still with me after those rollercoasters.
I can't image a better person to have in my life every day.

And finally, Henning. 
You can't read and, frankly, you can't really even sit yet. 
You may never see this statement.  
But I want you to know that seeing you smile when I get home every night has kept me going for the last four months. 
I can't wait to show you the world.

\end{abstract}
\cleardoublepage


%%%%%%%%%%%%%%%%%%%%%%%%%%%%%%%
% contribution statement
%%%%%%%%%%%%%%%%%%%%%%%%%%%%%%%
\chapter*{Author's Contributions}
This thesis covers a wide range of topics related to the search for tau neutrinos in DeepCore.
While I have personally accomplished much of the work of Chapters~\ref{chapter:vuvuzela}, \ref{chapter:muonsim}, \ref{chapter:greco}, and \ref{chapter:analysis}, I cannot claim sole credit for every task described.
Instead, I will outline my own contributions here.

In order to study backgrounds for my analysis, I spent a significant amount of my time working on simulations.
As part of my Master's degree at the University of Alabama, I studied the detector noise of IceCube in detail. 
Chapter~\ref{chapter:vuvuzela} represents a continuation of this work.
The creation and fitting of the Vuvuzela V2 noise model was performed by me at the Niels Bohr Institute during the winter of 2014-2015 without outside contributions.

The work described in Chapter~\ref{chapter:muonsim} is less clear.
I am not the primary author of the MuonGun or CORSIKA software packages in IceCube.
Instead, my work producing MuonGun simulation began as a partnership with Melanie Day, a postdoctoral reseacher at the University of Wisconsin who started preliminary tests with the generator in the spring of 2015.
My contributions to the DeepCore MuonGun sets (Section~\ref{sec:muongun_deepcore}) consisted of tuning the simulation scheme to more effiiciently use the collaboration's GPU resources.
I also chose the energy range and produced all MuonGun sets of Table~\ref{table:mgsets}.
As of the time of this writing, these are the only MuonGun simulation sets designed for oscillation analyses.

My work further improving the atmospheric muon simulation, described in Section~\ref{sec:kde_filtering}, was completed more recently.
The coding and testing were performed in the summer and fall of 2017.
I was able to combine the MuonGun code written by the IceCube collaboration with a general-purpose kernal density estimator from the SciPy python software package.
I am not the author of either of these projects, although the novel work presented in Section~\ref{sec:kde_filtering} was performed solely by me.

The GRECO event selection, discussed in Chapter~\ref{chapter:greco}, consists of a number of levels.
The DeepCoreFilter, described in Section~\ref{sec:DeepCoreFilter}, is my own project, but was written many years prior to my employment at the Niels Bohr Institute during my undergraduate degree at Penn State University.
The NoiseEngine project mentioned briefly in Section~\ref{subsec:level3_noise} is a contribution of mine from my master's degree work at the University of Alabama.
Neither of these algorithms was changed by me during the course of my PhD work.

The cuts included in the GRECO selection up to Level 3 (Section~\ref{sec:level3} were original written for the low energy working group in IceCube and I did not modify these.
The cuts performed at Level 4 (Section~\ref{sec:level4}) were written originally by Jason Koskinen for a previous incarnation of this analysis which suffered from poor modeling of backgrounds.
I took over the analysis when I joined the Niels Bohr Institute in January of 2014 and chose to use Jason's cuts and BDT without retraining on newer simulation sets.

The GRECO Levels 5 (Section~\ref{sec:level5}) and 6 (Section~\ref{sec:level6}) are my own work, although the algorithms described were not written by me.
Instead, my contribution consisted of the choice and optimization of variables for these levels with the goal of reducing the atmospheric muon and accidental trigger rates in the sample.
This process occurred concurrently with the calibration of the Vuvuzela V2 model from the summer of 2014 until the spring of 2015 .

The PegLeg reconstruction described in Section~\ref{sec:level7} is the work of Martin Leuermann, a PhD student at RTWH Aachen.
Likewise, the cuts described in Section~\ref{subsec:other_l7_cuts} were also discovered by Martin and added to the standard GRECO processing.
Martin's work investigating disagreement between data and simulation during his analysis led to the discovery of the flaring DOMs described in Section~\ref{subsec:flaring_doms}.
The discovery of disagreement in the simulated SPE template (Section~\ref{fig:new_spe_templates}) was informed by investigations performed by me, although the new templates were found through the work of Martin Rongen at RTWH Aachen and Spencer Axani at MIT.
I do not claim credit for the hard work performed by these collaborators.
The remaining work described in Section~\ref{sec:level7} was performed by me.

The analysis work presented in Chapter~\ref{chapter:analysis} was part of a collaborative effort amongst many PhD students. 
The choice of binning for the appearance analysis was informed by work done by Elim Cheung of the University of Maryland.
Martin Leuermann, Elim, and I performed analyses of the GRECO dataset in parallel to measure the neutrino mass hierarchy, muon neutrino disappearance, and tau neutrino appearance respectively. 
We have worked together continuously for nearly three years.
The analyses performed have many similarities and have benefited from the regular dialog between the three of us.

The OscFit software used to perform the fits for the appearance analysis was originally written by Juan Pablo Yanez for his PhD thesis work at Humboldt-Universität zu Berlin with later updates by Andrii Terliuk at DESY and Joshua Hignight at Michigan State University.
Martin and I have worked together to further upgrade this software, implementing additional systematics and other improvements for oscillation measurements.
Elim developed an independent fitter and performed checks of the OscFit code, finding and elimating numerous minor errors previously included in the fitter implementation.

The final measurements of Martin's, Elim's, and my searches were performed simultaneously, but independently.
The checks described in Sections~\ref{sec:systematics} through \ref{section:tau_results} were performed by me specifically for the tau appearance analysis.
Results from each were checked across all three analyses to improve our understanding of the sample.


My work has been varied over the course of my time at the Niels Bohr Institute.
This thesis exists as a partial record of my involvement in IceCube.
I regret that timing contraints prevent me from remarking upon my contributions to trigger development, background and signal simulation, and reconstruction work for a planned upgrade of the IceCube detector that I have performed during my time at the Niels Bohr Institute.
Despite the omissions, I hope that this record provides a useful description for my own efforts during the last four years and context for this thesis.

\cleardoublepage
